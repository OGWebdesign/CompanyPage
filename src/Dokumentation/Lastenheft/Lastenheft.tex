\documentclass[a4paper,12pt]{article}
\usepackage[left=2cm, right=2cm, top=2cm, bottom=2cm]{geometry}
\usepackage{amsmath}
\usepackage{graphicx}
\usepackage{array}
\usepackage{booktabs}
\usepackage{hyperref}

\begin{document}

\title{Lastenheft}
\author{Maximilian Gorn, Nils Ole Otto / OG Webdesign}
\date{\today}
\maketitle

\tableofcontents
\newpage

\section{Einleitung}

\subsection{Projektbezeichnung}
Entwicklung von Websites, SPAs, Bots und KI-Schnittstellen.

\subsection{Auftraggeber}
Maximilian Gorn, Nils Ole Otto / OG Webdesign \\
Sachsendorfer Hauptstraße 36 \\
03046 Cottbus \\
E-Mail: info@og-webdesing.de \\
Telefon: +49 123 4567890

\subsection{Auftragnehmer}
[Name des Auftraggebers]

\subsection{Projektbeschreibung}
Ziel des Projekts ist die Entwicklung von Webanwendungen, Websites, SPAs, Bots und KI-Schnittstellen unter Verwendung moderner Technologien wie TypeScript, JavaScript, Haskell, C\#, Java, React, Node.js, Next.js und MongoDB.

\section{Zielsetzung}

\subsection{Projektziele}
\begin{itemize}
    \item Entwicklung von modernen, benutzerfreundlichen Webseiten
    \item Erstellung von Single Page Applications (SPAs) für eine nahtlose Benutzererfahrung
    \item Entwicklung von Bots für verschiedene Anwendungsfälle
    \item Integration von KI-Schnittstellen in Webseiten zur Verbesserung der Funktionalität und Nutzererfahrung
\end{itemize}

\subsection{Muss-Anforderungen}
\begin{itemize}
    \item Verwendung von TypeScript, JavaScript, Haskell, C\#, Java, React, Node.js, Next.js und MongoDB
    \item Unterstützung für Responsive Design
    \item Einhaltung von Sicherheitsstandards
    \item Dokumentation und Git-Versionskontrolle
    \item Optimierte Performance und Skalierbarkeit
\end{itemize}

\subsection{Kann-Anforderungen}
\begin{itemize}
    \item Bereitstellung von zusätzlichen Dokumentationen und Tutorials
    \item Integration von Drittanbieter-APIs
    \item Mehrsprachigkeit der Anwendungen
\end{itemize}

\section{Produktübersicht}

\subsection{Produktfunktionen}
\begin{itemize}
    \item \textbf{Webseiten}: Moderne und interaktive Webseiten unter Verwendung von HTML, CSS, JavaScript und React.
    \item \textbf{SPAs}: Single Page Applications mit React und Next.js für eine dynamische Nutzererfahrung.
    \item \textbf{Bots}: Entwicklung von Bots für verschiedene Plattformen und Anwendungen.
    \item \textbf{KI-Schnittstellen}: Integration von KI-Funktionen in Webseiten zur Verbesserung der Benutzerinteraktion.
\end{itemize}

\subsection{Technologie-Stack}
\begin{itemize}
    \item \textbf{Frontend}: TypeScript, JavaScript, React, Next.js
    \item \textbf{Backend}: Node.js, C\#, Java
    \item \textbf{Datenbanken}: MongoDB
    \item \textbf{Sonstiges}: Git für Versionskontrolle, Haskell für spezielle Anforderungen
\end{itemize}

\section{Anforderungen}

\subsection{Funktionale Anforderungen}
\begin{itemize}
    \item \textbf{Benutzerregistrierung und -authentifizierung}: Sichere Registrierung und Anmeldung der Nutzer.
    \item \textbf{Datenverwaltung}: CRUD-Operationen (Create, Read, Update, Delete) für Daten.
    \item \textbf{Suchfunktion}: Effiziente Suche innerhalb der Anwendung.
    \item \textbf{Interaktive Benutzeroberflächen}: Reaktive und dynamische UI-Komponenten.
\end{itemize}

\subsection{Nicht-funktionale Anforderungen}
\begin{itemize}
    \item \textbf{Performance}: Schnelle Ladezeiten und reaktionsschnelle Benutzeroberflächen.
    \item \textbf{Sicherheit}: Schutz vor gängigen Sicherheitsbedrohungen wie XSS, CSRF und SQL Injection.
    \item \textbf{Skalierbarkeit}: Fähigkeit, bei steigender Nutzerzahl und Datenmengen zu skalieren.
    \item \textbf{Usability}: Intuitive und benutzerfreundliche Oberfläche.
\end{itemize}

\section{Projektmanagement}

\subsection{Projektphasen}
\begin{itemize}
    \item \textbf{Planung}: Anforderungsanalyse, Zeit- und Ressourcenplanung.
    \item \textbf{Design}: Erstellung von Wireframes, Mockups und Prototypen.
    \item \textbf{Entwicklung}: Implementierung der Anforderungen in der jeweiligen Technologie.
    \item \textbf{Testing}: Funktionale und nicht-funktionale Tests.
    \item \textbf{Deployment}: Veröffentlichung der Anwendung.
    \item \textbf{Wartung}: Kontinuierliche Pflege und Updates der Anwendung.
\end{itemize}

\subsection{Meilensteine}
\begin{itemize}
    \item \textbf{Meilenstein 1}: Abschluss der Anforderungsanalyse und Planung
    \item \textbf{Meilenstein 2}: Fertigstellung des Designs
    \item \textbf{Meilenstein 3}: Abschluss der Entwicklung
    \item \textbf{Meilenstein 4}: Abschluss der Testphase
    \item \textbf{Meilenstein 5}: Deployment der Anwendung
    \item \textbf{Meilenstein 6}: Abschluss der Wartungsphase
\end{itemize}

\subsection{Risikoanalyse}
\begin{itemize}
    \item \textbf{Technische Risiken}: Komplexität der Technologien, Integrationsprobleme.
    \item \textbf{Zeitliche Risiken}: Verzögerungen bei der Entwicklung und Lieferung.
    \item \textbf{Personelle Risiken}: Engpässe bei Personalressourcen.
\end{itemize}

\section{Abnahme}

\subsection{Abnahmekriterien}
\begin{itemize}
    \item Erfüllung aller Muss-Anforderungen
    \item Erfolgreiche Durchführung der Testfälle
    \item Positive Rückmeldungen der Benutzer
\end{itemize}

\subsection{Abnahmeprozess}
\begin{itemize}
    \item Regelmäßige Reviews und Feedback-Runden
    \item Abnahmetests durch den Auftraggeber
    \item Schriftliche Abnahmebestätigung
\end{itemize}

\section{Anhang}

\subsection{Glossar}
\begin{itemize}
    \item \textbf{CRUD}: Create, Read, Update, Delete
    \item \textbf{SPA}: Single Page Application
    \item \textbf{KI}: Künstliche Intelligenz
\end{itemize}

\subsection{Referenzen}
\begin{itemize}
    \item \href{https://reactjs.org/docs/getting-started.html}{Dokumentation zu React}
    \item \href{https://nodejs.org/en/docs/}{Dokumentation zu Node.js}
    \item \href{https://docs.mongodb.com/}{Dokumentation zu MongoDB}
    \item \href{https://nextjs.org/docs}{Dokumentation zu Next.js}
\end{itemize}

\end{document}
