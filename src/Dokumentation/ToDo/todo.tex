\documentclass{article}
\usepackage{enumitem}
\usepackage{geometry}

\geometry{a4paper, margin=1in}

\title{OTTO und GORN Webseite}
\author{}
\date{}

\begin{document}

\maketitle

\section*{Noch zu erledigen:}

\begin{enumerate}
    \item \textbf{Text mit Konzept und Sinn}
    \begin{itemize}
        \item Textinhalte erstellen
        \item Konzept und Sinn klar herausarbeiten
    \end{itemize}

    \item \textbf{Verlinkung mit Social Media und Einrichtung der Accounts}
    \begin{itemize}
        \item Social Media Accounts erstellen (Facebook, Twitter, Instagram, etc.)
        \item Links zu den Social Media Profilen auf der Webseite einfügen
    \end{itemize}

    \item \textbf{Robots.txt erstellen}
    \begin{itemize}
        \item Datei robots.txt anlegen
        \item Regeln für Suchmaschinen-Crawler festlegen
    \end{itemize}

    \item \textbf{Helmet und Conicals anlegen}
    \begin{itemize}
        \item Meta-Tags mit Helmet in React Framework hinzufügen
        \item Canonical URLs festlegen
    \end{itemize}

    \item \textbf{Fake URL Verlinkung mit Router Framework für das Crawling}
    \begin{itemize}
        \item Fake URLs für Crawling-Zwecke erstellen (SPA typisch)
        \item Routing im Framework (z.B. React Router) einrichten
    \end{itemize}

    \item \textbf{Reference Verweisungen auf die Seitenkomponenten}
    \begin{itemize}
        \item Referenzen zu einzelnen Komponenten auf der Webseite erstellen
    \end{itemize}

    \item \textbf{Kalkulator fertig stellen}
    \begin{itemize}
        \item Kalkulator-Feature entwickeln und testen
    \end{itemize}

    \item \textbf{Preisüberarbeitung und Preisentwicklung}
    \begin{itemize}
        \item Preise überprüfen und anpassen
        \item Entwicklung einer Preisstrategie
    \end{itemize}

    \item \textbf{Konkurrenzspionage um eventuelle Lücken zu finden für das Marketing}
    \begin{itemize}
        \item Wettbewerbsanalyse durchführen
        \item Marketinglücken identifizieren
    \end{itemize}

    \item \textbf{Steuer ID beantragen}
    \begin{itemize}
        \item Umsatzsteuer-Identifikationsnummer (USt-IdNr.)
    \end{itemize}

    \item \textbf{Genaue Besprechung und Konzeption der Pakete}
    \begin{itemize}
        \item Inhalte der Pakete zusammenstellen
        \item Produkte und Dienstleistungen beschreiben
    \end{itemize}

    \item \textbf{Die ganze Webseite muss einen roten Faden haben, der auf unser Marketing abbildet!}
    \begin{itemize}
        \item Gesamtkonzept der Webseite entwickeln
        \item Marketingstrategie in die Webseite integrieren
    \end{itemize}
\end{enumerate}

\end{document}
